\chapter{Sampling}

\section{Population, Sample, Sampling Design, and Inference}

\noindent \textbf{Population} – in a statistical study, the entire group of individuals about which we want information.

\vspace{0.2cm}

\noindent \textbf{Sample} – a part of the population from which we actually collect information. It is the individuals from whom information was gathered. We use a sample to draw conclusions about the population.

\vspace{0.2cm}

\noindent \textbf{Sampling design} – describes exactly how to choose a sample from the population

\vspace{0.2cm}

\noindent \textbf{Inference} – the process of drawing conclusions about the population from sample information.

\subsubsection*{Example 1}
Suppose we want to determine how much money UNM undergraduates spend on food every day. We selected 4,500 undergraduates and we determined that these students spend an average of \$20 a day on food. Determine the sample, population, and the inference. 

\vspace{0.2cm}

\textbf{Solution}

\vspace{0.2cm}

\noindent \textbf{Population}: UNM undergraduates

\vspace{0.2cm}

\noindent \textbf{Sample}: 4,500 undergraduates

\vspace{0.2cm}

\noindent \textbf{Inference}: UNM undergraduates spend an average of \$20 a day on food

\section{Bias}
A biased study design is one that systematically favors certain outcomes. 

\subsection{Undercoverage bias}
The sample is selected such that only a fraction of the individuals qualified to be in the sample get selected. There may be valid reasons why this kind of bias occurs.  

\subsection{Wording bias/effects}
The survey question is worded so that respondents give answers that favor a particular conclusion. 

\subsubsection*{Example 1}
A Gallup poll sponsored by the disposable diaper industry asked, "It is estimated that disposable diapers account for less than 2\% of the trash in today's landfills. In contrast, beverage containers, third-class mail and yard waste are estimated to account for about 21\% of the trash in landfills. Given this, in your opinion, would it be fair to ban disposable diapers?" 84\% responded No (Data from EESEE.) Which type of bias does this poll suffer from?

\vspace{0.2cm}

\textbf{Solution}

\vspace{0.2cm}

\noindent Wording bias/effect. The wording of this question is clearly in favor of the disposable diaper industry.

\subsection{Nonresponse bias}
Some of the individuals selected as part of the sample do not complete the survey. 

\subsubsection*{Example 1}
In a survey about a new immigration law in Georgia it is stated, ``46\% of the 132 Georgia farmers, agricultural processors, and farm service businesses who responded said they were experiencing some degree of labor shortage." (Al Hackle, "Groups seek reform, not repeal," Statesboro Herald, June 23, 2011). In interpreting the results of this survey, we should be wary of what type of bias?

a. Undercoverage bias

b. Nonresponse bias

c. Response bias

d. Question wording bias

e. Interviewer bias

\vspace{0.2cm}

\noindent \textbf{Solution}

\vspace{0.2cm}

\textbf{b. Nonresponse bias}

\subsection{How to deal with bias}
Choose the sample by randomness. Choosing a sample by randomness attacks bias by giving all individuals an equal chance to be chosen.


\section{Types of Sampling Design}

\subsection{Convenience Sample}
A sample selected by taking the members of the population that are easiest to reach. 

\subsubsection*{Example 1}
If I want to determine how much money UGA undergraduates spend on food every day, and I select only students in my class. This is a convenience sample.

\subsection{Voluntary Response Sample}
It consists of people who choose themselves by responding to a broad appeal. 

\subsubsection*{Example 1}
The issue of abortion is a hot topic right now in the US. Suppose the federal government is interested in knowing what the majority of Americans think about legalizing abortion. The government decides to set up a website to take the views of Americans. To get your views to the government, you will have to visit the website and complete a questionnaire. The sampling method used was a \textbf{voluntary} response sample.

\subsection{Voluntary Response Sample versus Convenience Sample}
In voluntary response sample, people choose whether to respond. In a convenience sample, the interviewer makes the choice.

\subsection{Simple Random Sample}
For a simple random sample (SRS),
\begin{itemize}
    \item every member of the population has an equal chance of being chosen.
    \item every possible sample has an equal chance of being chosen.
\end{itemize}

\noindent The key thing about SRS is that no preference is given to any individual or group.

\subsubsection*{Example 1}
Suppose you want to select a sample of 5 students from a class. To do the selection by \textbf{simple random sampling}, we will write the names of all the students in the class on slips. Then put the slips in a hat, and then draw out the names from the hat.

\subsection{Stratified Sampling}
To use stratified sampling, you have to classify the population into groups of similar individuals called strata. Then choose a separate SRS in each stratum and combine the SRSs to form the full sample. Stratified sampling ensures that each segment of the population is represented.

\subsubsection*{Example 1}
Suppose we want to select a sample of 5 students from a class by way of stratified random sampling. We will first divide the class into two similar groups (strata): males and females. Then we can use SRS to select 3 female students, and 2 male students. 






\chapter{Regression}

\section{What is Regression?}
Regression line is a straight line that describes how a response variable $y$ changes as the explanatory variable $x$ changes.

We often use a regression line to predict the value of $y$ for a given value of $x$ when the relationship between $x$ and $y$ is linear.

\vspace{0.2cm}

In this class, we are using the notation: $y=a+bx$ where $b =$ \textbf{slope} (change in $y$ divided by change in $x$, rise over run), and $a =$ \textbf{intercept} (the value of $y$ when $x=0$).

Note: In regression, we usually interpret $b$ (slope) as the average change in $y$ for one unit change in $x$. 


\subsubsection*{Example 1}
The equation below is a regression line of the relationship between a car’s highway gas mileage and its city gas mileage (in miles per gallon, mpg):

\[
\text{highway mpg} = 6.785 + (1.033 \times \text{city mpg})
\]

\noindent a. What is the slope of this line? Interpret the slope. 

\noindent b. What is the intercept? Interpret the intercept. 

\noindent c. Find the predicted highway mileage for a car that gets 16 miles per gallon in the city. Do the same for a car with city mileage of 28 mpg.  

\vspace{0.2cm}

\textbf{Solution}

\vspace{0.2cm}

\noindent a. The slope is 1.033. It means the average change in highway mpg for one unit change in city mpg is 1.033.

\vspace{0.2cm}

\noindent b. The intercept is 6.785. It means that when city mpg = 0, then highway mpg = 6.785. 

\vspace{0.2cm}

\noindent c. For 16 miles per gallon in the city, we have city mpg = 16. It follows that
\[
\text{highway mpg} = 6.785 + 1.033(16) = 6.785 + 16.528 = 23.31
\]

For 28 miles per gallon in the city, we have city mpg = 28. It follows that

\[
\text{highway mpg} = 6.785 + 1.033(28) = 6.785 + 28.924 = 35.71
\]


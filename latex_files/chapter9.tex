\chapter{Experiment}

In an \textbf{observational study}, we observe individuals (subjects) and measure variables of interest, but we do not attempt to influence the responses.

\begin{enumerate}
    \item The purpose of an observational study is to \textbf{describe some group or situation}.
    \item An observational study is a poor way to gauge the effect of a treatment.
\end{enumerate}

An \textbf{experiment}, on the other hand, deliberately imposes some \textbf{treatment} on individuals (subjects) to observe their responses. In a well-designed experiment, \textbf{treatments are randomly assigned to the individuals}. The purpose of an experiment is to \textbf{study whether the treatment causes a change in the response}.

\section{Subjects, Factors, Treatments and Placebo}

\textbf{Subjects:} The individuals (people) being examined (studied) in an observational study, or in an experiment.

\textbf{Factors:} The predictor/explanatory variables in a study.

\textbf{Treatment:} Any specific experimental condition applied to the subjects. \textbf{Treatments should be randomly assigned to subjects to avoid bias.} If an experiment has more than one factor, a treatment is a combination of specific values/levels of each factor.

\subsection*{Example 1}

Adults aged 60 to 83 years old were asked whether they had music lessons in their youth. In addition, each took a memory test and a brain function test. Scores on those tests were compared for those who took music lessons and those who didn't.

\begin{enumerate}
    \item What is/are the subject(s) in this study?
    \item What is/are the factor(s)?
    \item What is/are the response variable(s)?
    \item What type of study is this?
\end{enumerate}

\subsection*{Solution}

\begin{enumerate}
    \item Adults aged 60 to 83 years old.
    \item Whether or not you had music lessons in your youth.
    \item Test scores in the memory and brain function tests.
    \item Observational study.
\end{enumerate}


\subsection*{Example 2}

Two hair products, a styling cream and a mousse, were compared to see which gave the better curl hold. Forty women aged 18 to 25 were randomly split into two groups; one group was assigned to use the styling cream for the next two weeks; the other group was assigned to use the mousse. At the end of two weeks, each young woman reported on how well the product held their curl on a scale from 0 to 10 with 0 being the worst and 10 being the best you could possibly expect.

\begin{enumerate}
    \item What is/are the subject(s) in this study?
    \item What is/are the factor(s)?
    \item What type of study is this?
    \item What is/are the treatments?
    \item What is/are the response variable(s)?
\end{enumerate}

\subsection*{Solution}

\begin{enumerate}
    \item Forty women aged 18 to 25.
    \item Styling cream and Mousse.
    \item Experiment.
    \item Styling cream and Mousse.
    \item Ratings of how well the product held curl.
\end{enumerate}

\subsection*{Example 3}

An experiment is conducted to find out whether the color of your shoe (black, red, white) and your phone (iPhone, Android) can determine your score in an exam.

\begin{enumerate}
    \item What is/are the factor(s) in this study?
    \item What is/are the treatments in this study?
\end{enumerate}

\subsection*{Solution}

\begin{enumerate}
    \item There are 2 factors: Shoe Color and Phone Type.
    \item There are 6 treatments: the combination of the three levels of Shoe Color (black, red, white) and two levels of Phone Type (iPhone, Android).
\end{enumerate}

That is, $3 \times 2 = 6$ treatments.

 
\section{Double-Blind Experiments}

In a \textbf{double-blind} experiment, neither the subjects nor the people who interact with them know which treatment each subject is receiving. \textbf{Double-blind experiment reduces bias.}

\textbf{Placebo:} a dummy treatment.

\subsection*{Example 1}

Does drinking Echinacea tea reduce the duration of the common cold better than drinking tea without Echinacea? In a study to answer this question, 90 participants were randomly assigned to two groups with 45 in each. The first group drank tea with Echinacea and the second group drank tea without Echinacea. All teas had mint added to make them taste the same. In addition, the doctor diagnosing the length of their colds did not know who drank tea with Echinacea and who didn't.

\begin{enumerate}
    \item What is the placebo in this study?
    \item What other thing could be said about this study?
\end{enumerate}

\subsection*{Solution}

\begin{enumerate}
    \item The tea without Echinacea. Note that mint was added to both teas so you could not tell which of them had Echinacea.
    \item Double-blind experiment. Because both the subjects and the doctor did not know who drank tea with Echinacea and who didn't.
\end{enumerate}

\section{Confounding}

Two variables (explanatory variables or lurking variables) are \textbf{confounded} when their effects on a response variable cannot be distinguished from each other. That is, the effect of one variable is mixed up with the effect of the other variable.

\subsection*{Example 1}

It is known that when you exercise you sweat; and so, you lose water from your body when you exercise. Suppose we conduct a study in the summer to find out whether exercising makes you want to drink more water than usual. However, during the summer, people would want to drink more water because of the weather. Since the study was conducted during the summer, then the effect of the summer weather on you wanting to drink more water will be \textbf{confounded} (mixed up) with the effect of exercising.

Note that in this example, the \textbf{summer weather} is a \textbf{lurking variable}.

\subsection*{Example 2}

In an experiment for a new drug, to determine the most effective dose and method of administration, patients were randomly assigned to a 5-, 10-, 15-, or 20mg dose of the drug. In addition, the method of delivery of the drug (pill, skin patch, or nasal mist) was considered. In this experiment,

\begin{enumerate}
    \item How many factors were there? \textcolor{red}{2}
    \item How many treatments were there? \textcolor{red}{$4 \times 3 = 12$ treatments}
\end{enumerate}


\section{Completely Randomized Design (CRD)}

In a completely randomized experimental design, all the subjects are allocated at random among all the treatments.

\subsection*{Example 1}
Two hair products, a styling cream and a mousse, were compared to see which gave the better curl hold. Forty women aged 18 to 25 were randomly split into two groups; one group was assigned to use the styling cream for the next two weeks; the other group was assigned to use the mousse. At the end of two weeks, each young woman reported on how well the product held their curl on a scale from 0 to 10 with 0 being the worst and 10 being the best you could possibly expect.

\section{Block Design}

A \textbf{block} is a group of individuals that are known before the experiment to be similar in some way that is expected to affect the response to the treatments.

In a block design, the random assignment of individuals to treatments is carried out separately within each block.

\textbf{Note:} A Block Design removes variation associated with the blocking variable making the results of the experiment more precise than a Completely Randomized Design (CRD). 

\subsection*{Example 1}

Suppose we are interested in finding out whether Adidas or Nike shoes are the best for athletes (sprinters). We select 40 athletes (20 females and 20 males) for this study. Among the 20 females, we randomly give each athlete either a Nike shoe or an Adidas shoe. Among the 20 males, we randomly give each athlete either a Nike shoe or an Adidas shoe. Each of the 40 athletes runs a 100m race and we note the time it took them to cross the finish line.

\section{Matched Pairs}

A \textbf{Matched Pairs} design compares two treatments where the subjects are matched as closely as possible. Conducting an experiment involving twins, left leg and right leg, left eye and right eye, etc. are examples of matched pairs.

\subsection*{Example 1}

To compare the effectiveness of two detergents at removing common stains (such as mustard, grass, and blood), researchers prepared stained clothing samples. Each sample was cut in half and each of the two pieces was assigned, using a random mechanism, to one of the two detergents. After washing, the two halves were compared with each other.

Which term best describes the experimental design used in this study?

\begin{itemize}
    \item \textcolor{red}{a matched pairs experiment}
    \item a blocked experiment
    \item a completely randomized experiment
\end{itemize}

\subsection*{Example 2}

To determine if living next to high-voltage power lines increases the chance of getting cancer, researchers selected several homes at random, and then determined whether each home was within 50 yards of a high-voltage power line and whether anyone in the home had cancer. They compared the proportion of cancer cases in homes within 50 yards of a high-voltage power line with the proportion of cancer cases in homes more than 50 yards from a high-voltage power line. This is:

\begin{itemize}
    \item an experiment, but not a double-blind experiment.
    \item a matched pairs experiment.
    \item \textcolor{red}{an observational study.}
\end{itemize}

\subsection*{Example 3}

Suppose we are interested in finding out whether Adidas or Nike shoes are the best for athletes (sprinters). We select 40 athletes, and we randomly split them into two groups; one group is given Nike shoes; the other group is given Adidas shoes. Each athlete runs a 100m race and we note the time it took them to cross the finish line. Is this a completely randomized design or a block design?

\textcolor{red}{Completely randomized design}